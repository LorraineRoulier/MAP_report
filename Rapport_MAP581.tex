\documentclass[a4paper,12pt]{article}

\usepackage[latin1]{inputenc} % accents
\usepackage[T1]{fontenc}      % caract�res fran�ais
\usepackage{geometry}         % marges
\usepackage{lmodern}
\usepackage[french]{babel}
\usepackage{url,csquotes}
\usepackage[hidelinks,hyperfootnotes=false]{hyperref}
\usepackage{graphicx}
\usepackage[titlepage]{polytechnique}
\usepackage{textcomp}
\usepackage{float}
\usepackage{enumerate}
\usepackage{enumitem}%textbullets
 \frenchbsetup{StandardLists=true}%textbullets
 \usepackage{soul}
 \usepackage{color}
 \usepackage{amsfonts}
 
 
\title{Low Dimensional Embedding of Environmental Variables}      % renseigne le titre
\subtitle{EA MAP581}
\author{Flore Martin and Lorraine Roulier}           %   "   "   l'auteur
%\date{\today}           %   "   "   la future date de parution
\renewcommand{\thesection}{\arabic{section}}

\definecolor{bleu}{rgb}{0.5, 1.0, 1.0}
\newcommand{\hlb}[1]{\sethlcolor{bleu}\hl{#1}}


\begin{document}

\maketitle
\tableofcontents
\newpage

\section{Introduction}

Climate data amounts very quickly to a lot of unused data. In a day, we can collect temperature, pressure, wind data all over the world with satellites, even hourly. Our project was two sided. First, we familiarized with various dimension reduction techniques, then we attempted to show that the geographical position of a point on the planet - e.g. it's latitude and longitude - were embedded in the climate data one could gather on it. 

Dimension reduction techniques can be divided in two classes, linear dimension reduction and non linear dimension reduction. However, in all methods, the main goal is to figure out a similarity function between vectors. Such a function will then enable to sort the dataset into classes of vectors with similar features, which would have been more intricate with the initial dataset. 
We used a set of datasets we found on the NASA website, that gathered various means on climate variables over 22 years at every given latitude and longitude. These variables are gathered in the table below

\begin{figure}[H]
 \begin{center}
	
	\begin{tabular}{|c|c|c|c|c|}
		 \hline    
	  	Temperature & Pressure&  Relative Humidity& Wind Speed& Radiation\\ 
		
		�$C$ & $kPa$ & $ \%$ & $m/s$ & $kWh/m^{2}/day$ \\ 
		\hline
		  \end{tabular}
\end{center}
\caption{First lign of our dataset}
\end{figure}


\section{Principal Component Analysis - PCA}
	\subsection{Method}
	
		Principal Component Analysis detects tendencies in the data by maximizing the variance of the dataset matrix. This yields an orthonormal matrix that can be diagonalized. The largest eigenvalues point to the eigenvectors that contain the most information about the dataset. 
		
		Let $ X \in \mathbb{R}^{d \times n} $ be our dataset, PCA maximizes the following equation :
%%
\[ \| X - MM^{T}X \|^{2} \]
subject to $ M \in \mathcal{O}^{d \times r} $ where $ r<d $.

	\subsection{Results}
\section{Kernel Principal Component Analysis}
	\subsection{Method}
	\subsection{Results}
\section{Multidimensional Scaling - MDS}
	\subsection{Method}
	\subsection{Results}
\section{Isomap}
	\subsection{Method}
	\subsection{Results}
\section{Comparing the different methods}
\section{Conclusion}
\section{Bibliography}


\end{document}
